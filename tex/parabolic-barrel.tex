\documentclass{article}
\usepackage{tikz}
\usepackage{tikz-3dplot}
\usetikzlibrary{math}
\usepackage[active,tightpage]{preview}
\PreviewEnvironment{tikzpicture}
\setlength\PreviewBorder{1pt}
%
% File name: parabolic-barrel.tex
% Description: 
% A geometric representation of a parabolic barrel is shown.
% 
% Date of creation: December, 11th, 2020.
% Date of last modification: August, 19th, 2023.
% Author: Efraín Soto Apolinar.
% https://www.aprendematematicas.org.mx/author/efrain-soto-apolinar/instructing-courses/
% Source: Illustrated Dictionary of Mathematical Concepts
% https://www.aprendematematicas.org.mx/obras-distribucion-gratuita/
%
% Terms of use:
% According to TikZ.net
% https://creativecommons.org/licenses/by-nc-sa/4.0/
% Your commitment to the terms of use is greatly appreciated.
%
\begin{document}
	%
	\tdplotsetmaincoords{60}{130}
	\begin{tikzpicture}[tdplot_main_coords]
		\tikzmath{function f(\x) {return -0.125*(\x+\h)^2+0.6*(\x+\h)+1.0;};}
		\pgfmathsetmacro{\a}{-1.5}
		\pgfmathsetmacro{\b}{1.5}
		\pgfmathsetmacro{\h}{2.5}
		\pgfmathsetmacro{\xi}{\a}
		\pgfmathsetmacro{\domi}{\a-0.5}
		\pgfmathsetmacro{\domf}{\b+0.5}
		\pgfmathsetmacro{\step}{(\b-\a)/50.0}
		\pgfmathsetmacro{\xs}{\xi+\step}
		\pgfmathsetmacro{\max}{2.75}
		\pgfmathsetmacro{\radioi}{f(\a)}
		\pgfmathsetmacro{\radiof}{f(\b)}
		\pgfmathsetmacro{\R}{f(0.0)}
		% Distances (indications)
		% Height
		\draw[dash dot dot] (0,\a,0) -- (\max+0.75,\a,0);
		\draw[white] (0,\a,\max-0.25) -- (0,\b,\max-0.25) node [red,midway,sloped,fill=white] {$y = a\,x^{2} + b\,x + c$};
		% Lid at y = a
		\fill[cyan,opacity=0.75] plot[domain=0.0:2.0*pi,smooth,variable=\t] ({\radioi*cos(\t r)},{\a},{\radioi*sin(\t r});	
		\draw[thick] (0,\a,0.0625) -- (0,\a,-0.0625);
		% Circumferences (behind the coordinate axis)
		\foreach \x in {\xi,\xs,...,\b}{
			\pgfmathsetmacro{\radio}{f(\x)}	% Radius of circle at start of solid of revolution
			\draw[cyan,very thick,opacity=0.35] plot[domain=0.5*pi:2.0*pi,smooth,variable=\t] ({\radio*cos(\t r)},{\x},{\radio*sin(\t r});	
		}
		% I draw the part of the solid of revolution that lies behind the coordinate axes
		\foreach \angulo in {358,356,...,90}{
			\draw[cyan,very thick,rotate around y=\angulo,opacity=0.35] plot[domain=\a:\b,smooth,variable=\t] ({0},{\t},{f(\t)});
		}
		% The graph of the function that is rotated about the y-axis
		\draw[red,ultra thick] plot[domain=\domi:\b,smooth,variable=\t] ({0},{\t},{f(\t)});
		% Coordinate axis
		\draw[thick] (0,\domi-1.0,0) -- (0,\b,0);
		\draw[thick,->] (0,0,0) -- (0,0,\max) node [above] {$y$};
		% Circumferences (part that is in front of the coordinate axes)
		\foreach \x in {\xi,\xs,...,\b}{
			\pgfmathsetmacro{\radio}{f(\x)}	% Radio del círculo al inicio del sólido de revolución
			\draw[cyan,very thick,opacity=0.35] plot[domain=0.0:0.5*pi,smooth,variable=\t] ({\radio*cos(\t r)},{\x},{\radio*sin(\t r});	
		}
		% The solid of revolution (the part facing the coordinate axes)
		\foreach \angulo in {0,2,...,89}{
			\draw[cyan,very thick,rotate around y=\angulo,opacity=0.35] plot[domain=\a:\b,smooth,variable=\t] ({0},{\t},{f(\t)});
		}
		% Lid at y = b
		\fill[cyan,opacity=0.75] plot[domain=0.0:2.0*pi,smooth,variable=\t] ({\radiof*cos(\t r)},{\b},{\radiof*sin(\t r});
		\draw[thick] (0,\b,0.0625) -- (0,\b,-0.0625);
		% y-axis (last part)
		\draw[thick,->] (0,\b,0) -- (0,\domf+2.0,0) node [right] {$x$};
		% plot of z = f(y) [last part]
		\draw[red,ultra thick,->] plot[domain=\b:\domf,smooth,variable=\t] ({0},{\t},{f(\t)});
		% Distances
		% Height of the barrel
		\draw[dash dot dot] (0,\b,0) -- (\max+0.75,\b,0);
		\draw[thick,<->] (\max+0.35,\a,0) -- (\max+0.35,\b,0) node [midway,sloped,fill=white] {$h$};
		% Minor radius (r)
		\draw[dash dot dot] (0,\a,\radioi) -- (\max+0.75,\a,\radioi);
		\draw[thick,<->] (\max+0.35,\a,0) -- (\max+0.35,\a,\radioi) node [midway,fill=white] {$r$};
		% Major radius (R)
		\draw[dash dot dot] (0,0,\R) -- (0,\b+1.75,\R);
		\draw[thick,<->] (0,\b+1.65,0) -- (0,\b+1.65,\R) node [midway,fill=white] {$R$};
	\end{tikzpicture}
\end{document}
