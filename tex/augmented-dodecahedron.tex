\documentclass{article}
\usepackage{tikz}
\usetikzlibrary{patterns}
%\usetikzlibrary{math}
\usepackage{tikz-3dplot}
\usepackage[active,tightpage]{preview}
\PreviewEnvironment{tikzpicture}
\setlength\PreviewBorder{0.125pt}
%
% File name: augmented-dodecahedron.tex
% Description: 
% A geometric representation of an augmented dodecahedron is shown.
% 
% Date of creation: June, 7th, 2021.
% Date of last modification: June, 7th, 2021.
% Author: Efraín Soto Apolinar.
% https://www.aprendematematicas.org.mx/author/efrain-soto-apolinar/instructing-courses/
% Source: Illustrated Dictionary of Mathematical Concepts
% https://www.aprendematematicas.org.mx/obras-distribucion-gratuita/
%
% Terms of use:
% According to TikZ.net
% https://creativecommons.org/licenses/by-nc-sa/4.0/
% Your commitment to the terms of use is greatly appreciated.
%
\begin{document}
\tdplotsetmaincoords{60}{120}
%
\begin{tikzpicture}[tdplot_main_coords,scale=1.75]
	% Coordinates of the vertices
	\coordinate(1) at (0., 0., 1.40126);
	\coordinate(2) at (0., 0., -1.40126);
	\coordinate(3) at (0.178411, -1.30902, 0.467086);
	\coordinate(4) at (0.178411, 1.30902, 0.467086);
	\coordinate(5) at (0.467086, -0.809017, -1.04444);
	\coordinate(6) at (0.467086, 0.809017, -1.04444);
	\coordinate(7) at (1.04444, -0.809017, 0.467086);
	\coordinate(8) at (1.04444, 0.809017, 0.467086);
	\coordinate(9) at (-1.22285, -0.5, 0.467086);
	\coordinate(10) at (-1.22285, 0.5, 0.467086);
	%
	\coordinate(11) at (1.22285, -0.5, -0.467086);
	\coordinate(12) at (1.22285, 0.5, -0.467086);
	\coordinate(13) at (-0.934172, 0., -1.04444);
	\coordinate(14) at (-0.467086, -0.809017, 1.04444);
	\coordinate(15) at (-0.467086, 0.809017, 1.04444);
	\coordinate(16) at (0.934172, 0., 1.04444);
	\coordinate(17) at (-1.04444, -0.809017, -0.467086);
	\coordinate(18) at (-1.04444, 0.809017, -0.467086);
	\coordinate(19) at (-0.995125, 0., 1.30264);
	\coordinate(20) at (-0.178411, -1.30902, -0.467086);
	%
	\coordinate(21) at (-0.178411, 1.30902, -0.467086);
	% Faces of the polyhedron
	\draw[red,thick,fill=cyan!35,opacity=0.75]  (2) -- (6) -- (12) -- (11) -- (5) -- cycle;
	\draw[red,thick,fill=cyan!35,opacity=0.75]  (11) -- (12) -- (8) -- (16) -- (7) -- cycle;
	\draw[red,thick,fill=cyan!35,opacity=0.75]  (3) -- (7) -- (16) -- (1) -- (14) -- cycle;
	\draw[red,thick,fill=cyan!35,opacity=0.75]  (16) -- (8) -- (4) -- (15) -- (1) -- cycle;
	\draw[red,thick,fill=cyan!35,opacity=0.75]  (19) -- (14) -- (1) -- cycle;
	\draw[red,thick,fill=cyan!35,opacity=0.75]  (19) -- (1) -- (15) -- cycle;
	\draw[red,thick,fill=cyan!35,opacity=0.75]  (12) -- (6) -- (21) -- (4) -- (8) -- cycle;
	\draw[red,thick,fill=cyan!35,opacity=0.75]  (19) -- (15) -- (10) -- cycle;
	\draw[red,thick,fill=cyan!35,opacity=0.75]  (4) -- (21) -- (18) -- (10) -- (15) -- cycle;
	%
	\draw[red,thick,fill=cyan!35,opacity=0.75]  (6) -- (2) -- (13) -- (18) -- (21) -- cycle;
	\draw[red,thick,fill=cyan!35,opacity=0.75]  (5) -- (11) -- (7) -- (3) -- (20) -- cycle;
	\draw[red,thick,fill=cyan!35,opacity=0.75]  (19) -- (9) -- (14) -- cycle;
	\draw[red,thick,fill=cyan!35,opacity=0.75]  (19) -- (10) -- (9) -- cycle;
	\draw[red,thick,fill=cyan!35,opacity=0.75]  (17) -- (20) -- (3) -- (14) -- (9) -- cycle;
	\draw[red,thick,fill=cyan!35,opacity=0.75]  (18) -- (13) -- (17) -- (9) -- (10) -- cycle;
	\draw[red,thick,fill=cyan!35,opacity=0.75]  (2) -- (5) -- (20) -- (17) -- (13) -- cycle;
	%
	\end{tikzpicture}
	%
\end{document}



