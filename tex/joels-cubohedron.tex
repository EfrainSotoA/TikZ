\documentclass{article}
\usepackage{tikz}
\usepackage{tikz-3dplot}
\usepackage[active,tightpage]{preview}
\PreviewEnvironment{tikzpicture}
\setlength\PreviewBorder{0.125pt}
%
% File name: joels-cubohedron.tex
% Description: 
% A geometric representation of a joel's cubohedron is shown.
% 
% Date of creation: July, 30th, 2021.
% Date of last modification: August, 19th, 2023.
% Author: Efraín Soto Apolinar.
% https://www.aprendematematicas.org.mx/author/efrain-soto-apolinar/instructing-courses/
% Source: Illustrated Dictionary of Mathematical Concepts
% https://www.aprendematematicas.org.mx/obras-distribucion-gratuita/
%
% Terms of use:
% According to TikZ.net
% https://creativecommons.org/licenses/by-nc-sa/4.0/
% Your commitment to the terms of use is greatly appreciated.
%
\begin{document}
\tdplotsetmaincoords{80}{115}
%
\begin{tikzpicture}[tdplot_main_coords]
	%
	\pgfmathsetmacro{\escala}{3.25}
	\pgfmathsetmacro{\a}{\escala}
	\pgfmathsetmacro{\d}{\a * (3 / 8)}
	% Coordinates of the vertices
	\coordinate(0) at (0,0,0);
	\coordinate(1) at (\a,0,0);
	\coordinate(2) at (\a,\a,0);
	\coordinate(3) at (0,\a,0);
	\coordinate(4) at (0,0,\a);
	\coordinate(5) at (\a,0,\a);
	\coordinate(6) at (\a,\a,\a);
	\coordinate(7) at (0,\a,\a);
	\coordinate(8) at (0.5*\a,0.5*\a,0.5*\a);
	\coordinate(9) at (0.5*\a,0.5*\a,\d); % z = 0
	\coordinate(10) at (0.5*\a,\d,0.5*\a); % y = 0
	\coordinate(11) at (\d,0.5*\a,0.5*\a); % x = 0
	\coordinate(12) at (0.5*\a,0.5*\a,\a-\d); % z = 1
	\coordinate(13) at (0.5*\a,\a-\d,0.5*\a); % y = 1
	\coordinate(14) at (\a-\d,0.5*\a,0.5*\a); % x = 1
	% Faces of the polyhedron
	% Face: x = 0
	\draw[red,thick,fill=cyan!35,opacity=0.75]  (0) -- (3) -- (11) -- cycle;
	\draw[red,thick,fill=cyan!35,opacity=0.75]  (3) -- (7) -- (11) -- cycle;
	\draw[red,thick,fill=cyan!35,opacity=0.75]  (7) -- (4) -- (11) -- cycle;
	\draw[red,thick,fill=cyan!35,opacity=0.75]  (4) -- (0) -- (11) -- cycle;
	% Face: y = 0
	\draw[red,thick,fill=cyan!35,opacity=0.75]  (0) -- (1) -- (10) -- cycle;
	\draw[red,thick,fill=cyan!35,opacity=0.75]  (5) -- (4) -- (10) -- cycle;
	\draw[red,thick,fill=cyan!35,opacity=0.75]  (4) -- (0) -- (10) -- cycle;
	\draw[red,thick,fill=cyan!35,opacity=0.75]  (1) -- (5) -- (10) -- cycle;
	% Face: z = 0
	\draw[red,thick,fill=cyan!35,opacity=0.75]  (3) -- (0) -- (9) -- cycle;
	\draw[red,thick,fill=cyan!35,opacity=0.75]  (0) -- (1) -- (9) -- cycle;
	\draw[red,thick,fill=cyan!35,opacity=0.75]  (1) -- (2) -- (9) -- cycle;
	\draw[red,thick,fill=cyan!35,opacity=0.75]  (2) -- (3) -- (9) -- cycle;
	% Face: z = 1
	\draw[red,thick,fill=cyan!35,opacity=0.75]  (4) -- (5) -- (12) -- cycle;
	\draw[red,thick,fill=cyan!35,opacity=0.75]  (7) -- (4) -- (12) -- cycle;
	\draw[red,thick,fill=cyan!35,opacity=0.75]  (6) -- (7) -- (12) -- cycle;
	\draw[red,thick,fill=cyan!35,opacity=0.75]  (5) -- (6) -- (12) -- cycle;
	% Face: y = 1
	\draw[red,thick,fill=cyan!35,opacity=0.75]  (2) -- (3) -- (13) -- cycle;
	\draw[red,thick,fill=cyan!35,opacity=0.75]  (3) -- (7) -- (13) -- cycle;
	\draw[red,thick,fill=cyan!35,opacity=0.75]  (7) -- (6) -- (13) -- cycle;
	\draw[red,thick,fill=cyan!35,opacity=0.75]  (6) -- (2) -- (13) -- cycle;
	% Face: x = 1
	\draw[red,thick,fill=cyan!35,opacity=0.75]  (1) -- (2) -- (14) -- cycle;
	\draw[red,thick,fill=cyan!35,opacity=0.75]  (2) -- (6) -- (14) -- cycle;
	\draw[red,thick,fill=cyan!35,opacity=0.75]  (6) -- (5) -- (14) -- cycle;
	\draw[red,thick,fill=cyan!35,opacity=0.75]  (5) -- (1) -- (14) -- cycle;
	%
	\end{tikzpicture}
	%
\end{document}

