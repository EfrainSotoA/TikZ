\documentclass{article}
\usepackage{tikz}
\usepackage{tikz-3dplot}
\usepackage[active,tightpage]{preview}
\PreviewEnvironment{tikzpicture}
\setlength\PreviewBorder{0.125pt}
%
% File name: augmented-truncated-cube.tex
% Description: 
% A geometric representation of an augmented truncated cube is shown.
% 
% Date of creation: June, 18th, 2021.
% Date of last modification: June, 18th, 2021.
% Author: Efraín Soto Apolinar.
% https://www.aprendematematicas.org.mx/author/efrain-soto-apolinar/instructing-courses/
% Source: Illustrated Dictionary of Mathematical Concepts
% https://www.aprendematematicas.org.mx/obras-distribucion-gratuita/
%
% Terms of use:
% According to TikZ.net
% https://creativecommons.org/licenses/by-nc-sa/4.0/
% Your commitment to the terms of use is greatly appreciated.
%
\begin{document}
\tdplotsetmaincoords{80}{115}
%
\begin{tikzpicture}[tdplot_main_coords,scale=1.5]
	% Coordinates of the vertices
	\coordinate(1) at (-0.5, 1.20711, -1.20711);
	\coordinate(2) at (-0.5, 1.20711, 1.20711);
	\coordinate(3) at (-0.5, -1.20711, -1.20711);
	\coordinate(4) at (-0.5, -1.20711, 1.20711);
	\coordinate(5) at (0., -0.707107, 1.91421);
	\coordinate(6) at (0., 0.707107, 1.91421);
	\coordinate(7) at (0.5, 1.20711, -1.20711);
	\coordinate(8) at (0.5, 1.20711, 1.20711);
	\coordinate(9) at (0.5, -1.20711, -1.20711);
	\coordinate(10) at (0.5, -1.20711, 1.20711);
	%
	\coordinate(11) at (-0.707107, 0., 1.91421);
	\coordinate(12) at (0.707107, 0., 1.91421);
	\coordinate(13) at (1.20711, -0.5, -1.20711);
	\coordinate(14) at (1.20711, -0.5, 1.20711);
	\coordinate(15) at (1.20711, 0.5, -1.20711);
	\coordinate(16) at (1.20711, 0.5, 1.20711);
	\coordinate(17) at (1.20711, 1.20711, -0.5);
	\coordinate(18) at (1.20711, 1.20711, 0.5);
	\coordinate(19) at (1.20711, -1.20711, -0.5);
	\coordinate(20) at (1.20711, -1.20711, 0.5);
	%
	\coordinate(21) at (-1.20711, -0.5, -1.20711);
	\coordinate(22) at (-1.20711, -0.5, 1.20711);
	\coordinate(23) at (-1.20711, 0.5, -1.20711);
	\coordinate(24) at (-1.20711, 0.5, 1.20711);
	\coordinate(25) at (-1.20711, 1.20711, -0.5);
	\coordinate(26) at (-1.20711, 1.20711, 0.5);
	\coordinate(27) at (-1.20711, -1.20711, -0.5);
	\coordinate(28) at (-1.20711, -1.20711, 0.5);
	% Faces of the polyhedron
	\draw[red,thick,fill=cyan!35,opacity=0.75]  (7) -- (15) -- (13) -- (9) -- (3) -- (21) -- (23) -- (1) -- cycle;
	\draw[red,thick,fill=cyan!35,opacity=0.75]  (9) -- (19) -- (20) -- (10) -- (4) -- (28) -- (27) -- (3) -- cycle;
	\draw[red,thick,fill=cyan!35,opacity=0.75]  (21) -- (27) -- (28) -- (22) -- (24) -- (26) -- (25) -- (23) -- cycle;
	\draw[red,thick,fill=cyan!35,opacity=0.75]  (27) -- (21) -- (3) -- cycle;
	\draw[red,thick,fill=cyan!35,opacity=0.75]  (1) -- (23) -- (25) -- cycle;
	\draw[red,thick,fill=cyan!35,opacity=0.75]  (4) -- (22) -- (28) -- cycle;
	\draw[red,thick,fill=cyan!35,opacity=0.75]  (26) -- (24) -- (2) -- cycle;
	\draw[red,thick,fill=cyan!35,opacity=0.75]  (11) -- (6) -- (2) -- (24) -- cycle;
	\draw[red,thick,fill=cyan!35,opacity=0.75]  (5) -- (11) -- (22) -- (4) -- cycle;
	\draw[red,thick,fill=cyan!35,opacity=0.75]  (24) -- (22) -- (11) -- cycle;
	\draw[red,thick,fill=cyan!35,opacity=0.75]  (4) -- (10) -- (5) -- cycle;
	%
	\draw[red,thick,fill=cyan!35,opacity=0.75]  (9) -- (13) -- (19) -- cycle;
	\draw[red,thick,fill=cyan!35,opacity=0.75]  (17) -- (15) -- (7) -- cycle;
	\draw[red,thick,fill=cyan!35,opacity=0.75]  (8) -- (2) -- (6) -- cycle;
	\draw[red,thick,fill=cyan!35,opacity=0.75]  (8) -- (16) -- (18) -- cycle;
	\draw[red,thick,fill=cyan!35,opacity=0.75]  (1) -- (25) -- (26) -- (2) -- (8) -- (18) -- (17) -- (7) -- cycle;
	\draw[red,thick,fill=cyan!35,opacity=0.75]  (20) -- (14) -- (10) -- cycle;
	\draw[red,thick,fill=cyan!35,opacity=0.75]  (15) -- (17) -- (18) -- (16) -- (14) -- (20) -- (19) -- (13) -- cycle;
	\draw[red,thick,fill=cyan!35,opacity=0.75]  (14) -- (16) -- (12) -- cycle;
	\draw[red,thick,fill=cyan!35,opacity=0.75]  (6) -- (12) -- (16) -- (8) -- cycle;
	\draw[red,thick,fill=cyan!35,opacity=0.75]  (12) -- (5) -- (10) -- (14) -- cycle;
	\draw[red,thick,fill=cyan!35,opacity=0.75]  (6) -- (11) -- (5) -- (12) -- cycle;
	%
	\end{tikzpicture}
	%
\end{document}



