\documentclass{article}
\usepackage{tikz}
\usepackage{tikz-3dplot}
\usepackage[active,tightpage]{preview}
\PreviewEnvironment{tikzpicture}
\setlength\PreviewBorder{0.125pt}
%
% File name: square-bipyramid.tex
% Description: 
% A geometric representation of a square bipyramid is shown.
% 
% Date of creation: May, 21st, 2021.
% Date of last modification: August, 19th, 2023.
% Author: Efraín Soto Apolinar.
% https://www.aprendematematicas.org.mx/author/efrain-soto-apolinar/instructing-courses/
% Source: Illustrated Dictionary of Mathematical Concepts
% https://www.aprendematematicas.org.mx/obras-distribucion-gratuita/
%
% Terms of use:
% According to TikZ.net
% https://creativecommons.org/licenses/by-nc-sa/4.0/
% Your commitment to the terms of use is greatly appreciated.
%
\begin{document}
%
\tdplotsetmaincoords{70}{140}
%
\begin{tikzpicture}[tdplot_main_coords]
	\pgfmathsetmacro{\r}{1.5}
	\pgfmathsetmacro{\L}{sqrt(2.0)*\r}
	\pgfmathsetmacro{\H}{0.5*sqrt(2.0)*\L}
	\pgfmathsetmacro{\n}{4.0}
	\pgfmathsetmacro{\anguloi}{60}
	\pgfmathsetmacro{\angulo}{360.0/\n}
	\pgfmathsetmacro{\criterio}{int(0.5*\n)}
	\pgfmathsetmacro{\pxn}{\r*cos(0.5*\anguloi)}
	\pgfmathsetmacro{\pyn}{\r*sin(0.5*\anguloi)}
	% Compute the coordinates of the vertices
	\foreach \vertice in {1,2,...,\n}{
		\pgfmathsetmacro{\pxi}{\r*cos((\angulo)*\vertice+\anguloi)}
		\pgfmathsetmacro{\pyi}{\r*sin((\angulo)*\vertice+\anguloi)}
		\coordinate (P\vertice) at (\pxi,\pyi,0);
	}
	% Draw the polyhedron
	% The tringular faces
	\foreach \vertice in {1,2,...,\n}{
		\pgfmathsetmacro{\vf}{\vertice+1}
		\ifthenelse{\vertice<\n}{
			\fill[cyan!50,opacity=0.5] (P\vertice) -- (P\vf) -- (0,0,\H) -- (P\vertice);
			\fill[cyan!50,opacity=0.5] (P\vertice) -- (P\vf) -- (0,0,-\H) -- (P\vertice);
			%
			\draw[red,thick] (P\vertice) -- (P\vf);
			\draw[red,thick] (P\vertice) -- (0,0,\H);
			\draw[red,thick] (P\vertice) -- (0,0,-\H);
		}{ % The last face
			\fill[cyan!50,opacity=0.5] (P\vertice) -- (P1) -- (0,0,\H) -- (P\vertice);
			\fill[cyan!50,opacity=0.5] (P\vertice) -- (P1) -- (0,0,-\H) -- (P\vertice);
			\draw[red,thick] (P\vertice) -- (P1);
			\draw[red,thick] (P\vertice) -- (0,0,\H);
			\draw[red,thick] (P\vertice) -- (0,0,-\H);
		}
	}
\end{tikzpicture}
%
\end{document}