\documentclass{article}
\usepackage{tikz}
\usetikzlibrary{patterns}
%\usetikzlibrary{math}
\usepackage{tikz-3dplot}
\usepackage[active,tightpage]{preview}
\PreviewEnvironment{tikzpicture}
\setlength\PreviewBorder{0.125pt}
%
% File name: snub-disphenoid.tex
% Description: 
% A geometric representation of a snub disphenoid is shown.
% 
% Date of creation: June, 2nd, 2021.
% Date of last modification: August, 19th, 2023.
% Author: Efraín Soto Apolinar.
% https://www.aprendematematicas.org.mx/author/efrain-soto-apolinar/instructing-courses/
% Source: Illustrated Dictionary of Mathematical Concepts
% https://www.aprendematematicas.org.mx/obras-distribucion-gratuita/
%
% Terms of use:
% According to TikZ.net
% https://creativecommons.org/licenses/by-nc-sa/4.0/
% Your commitment to the terms of use is greatly appreciated.
%
\begin{document}
\tdplotsetmaincoords{90}{150}
%
\begin{tikzpicture}[tdplot_main_coords]
	%
	\pgfmathsetmacro{\escala}{4.0}
	% Coordinates of the vertices
	\coordinate(1) at (-\escala*0.5, \escala*0., \escala*0.);
	\coordinate(2) at (\escala*0., -\escala*0.5, \escala*1.56786);
	\coordinate(3) at (\escala*0., \escala*0.5, \escala*1.56786);
	\coordinate(4) at (\escala*0., -\escala*0.644584, \escala*0.578369);
	\coordinate(5) at (\escala*0., \escala*0.644584, \escala*0.578369);
	\coordinate(6) at (\escala*0.5, \escala*0., \escala*0.);
	\coordinate(7) at (-\escala*0.644584, \escala*0., \escala*0.989492);
	\coordinate(8) at (\escala*0.644584, \escala*0., \escala*0.989492);
	% Faces of the polyhedron
	\draw[red,thick,fill=cyan!35,opacity=0.75]  (6) -- (8) -- (4) -- cycle; 
	\draw[red,thick,fill=cyan!35,opacity=0.75]  (4) -- (2) -- (7) -- cycle;
	\draw[red,thick,fill=cyan!35,opacity=0.75]  (7) -- (1) -- (4) -- cycle; 
	\draw[red,thick,fill=cyan!35,opacity=0.75]  (2) -- (3) -- (7) -- cycle; 
	\draw[red,thick,fill=cyan!35,opacity=0.75]  (4) -- (1) -- (6) -- cycle;
	\draw[red,thick,fill=cyan!35,opacity=0.75]  (2) -- (4) -- (8) -- cycle; 
	\draw[red,thick,fill=cyan!35,opacity=0.75]  (5) -- (6) -- (1) -- cycle;
	\draw[red,thick,fill=cyan!35,opacity=0.75]  (3) -- (2) -- (8) -- cycle;
	\draw[red,thick,fill=cyan!35,opacity=0.75]  (8) -- (5) -- (3) -- cycle; 
	\draw[red,thick,fill=cyan!35,opacity=0.75]  (5) -- (8) -- (6) -- cycle;
	\draw[red,thick,fill=cyan!35,opacity=0.75]  (3) -- (5) -- (7) -- cycle; 
	\draw[red,thick,fill=cyan!35,opacity=0.75]  (5) -- (1) -- (7) -- cycle; 
	%
	\end{tikzpicture}
	%
\end{document}



