\documentclass{article}
\usepackage{tikz}
\usepackage{tikz-3dplot}
\usepackage[active,tightpage]{preview}
\PreviewEnvironment{tikzpicture}
\setlength\PreviewBorder{0.125pt}
%
% File name: canonical-rectified-truncated-tetrahedron.tex
% Description: 
% A geometric representation of a canonical rectified truncated tetrahedron is shown.
% 
% Date of creation: July, 22nd, 2020.
% Date of last modification: August, 19th, 2023.
% Author: Efraín Soto Apolinar.
% https://www.aprendematematicas.org.mx/author/efrain-soto-apolinar/instructing-courses/
% Source: Illustrated Dictionary of Mathematical Concepts
% https://www.aprendematematicas.org.mx/obras-distribucion-gratuita/
%
% Terms of use:
% According to TikZ.net
% https://creativecommons.org/licenses/by-nc-sa/4.0/
% Your commitment to the terms of use is greatly appreciated.
%
\begin{document}
\tdplotsetmaincoords{70}{140}
%
\begin{tikzpicture}[tdplot_main_coords,scale=2.5]
	%
	\pgfmathsetmacro{\cero}{sqrt(23.0 * (2.0)^(1.0/3.0) - 3 * (4)^(1.0/3.0) - 15.0) / 11.0}
	\pgfmathsetmacro{\uno}{sqrt(65.0 - 19.0 * (2)^(1.0/3.0) + 13.0 * (4)^(1.0/3.0)) / 11.0}
	\pgfmathsetmacro{\dos}{sqrt(1.0 - (2.0)^(1.0/3.0) + (4.0)^(1.0/3.0))}
	% Coordenadas de los vértices
	\coordinate(0) at (0.0, 0.0, \dos);
	\coordinate(1) at (0.0, 0.0, -\dos);
	\coordinate(2) at (\dos, 0.0, 0.0);
	\coordinate(3) at (-\dos, 0.0, 0.0);
	\coordinate(4) at (0.0, \dos, 0.0);
	\coordinate(5) at (0.0, -\dos, 0.0);
	\coordinate(6) at (\uno, -\cero, \uno);
	\coordinate(7) at (\uno, \cero, -\uno);
	\coordinate(8) at (-\uno, \cero, \uno);
	\coordinate(9) at (-\uno, -\cero, -\uno);
	\coordinate(10) at (\uno, -\uno, \cero);
	\coordinate(11) at (\uno, \uno, -\cero);
	\coordinate(12) at (-\uno, \uno, \cero);
	\coordinate(13) at (-\uno, -\uno, -\cero);
	\coordinate(14) at (\cero, -\uno, \uno);
	\coordinate(15) at (\cero, \uno, -\uno);
	\coordinate(16) at (-\cero, \uno, \uno);
	\coordinate(17) at (-\cero, -\uno, -\uno);
	% Faces of the polyhedron
	\draw[red,thick,fill=cyan!35,opacity=0.75]  (0) -- (6) -- (2) -- (11) -- (4) -- (16) -- cycle;
	\draw[red,thick,fill=cyan!35,opacity=0.75]  (6) -- (14) -- (10) -- cycle;
	\draw[red,thick,fill=cyan!35,opacity=0.75]  (7) -- (15) -- (11) -- cycle;
	\draw[red,thick,fill=cyan!35,opacity=0.75]  (0) -- (14) -- (6) -- cycle;
	\draw[red,thick,fill=cyan!35,opacity=0.75]  (0) -- (16) -- (8) -- cycle;
	\draw[red,thick,fill=cyan!35,opacity=0.75]  (1) -- (15) -- (7) -- cycle;
	\draw[red,thick,fill=cyan!35,opacity=0.75]  (2) -- (6) -- (10) -- cycle;
	\draw[red,thick,fill=cyan!35,opacity=0.75]  (2) -- (7) -- (11) -- cycle;
	\draw[red,thick,fill=cyan!35,opacity=0.75]  (4) -- (11) -- (15) -- cycle;
	\draw[red,thick,fill=cyan!35,opacity=0.75]  (4) -- (12) -- (16) -- cycle;
	\draw[red,thick,fill=cyan!35,opacity=0.75]  (5) -- (10) -- (14) -- cycle;
	\draw[red,thick,fill=cyan!35,opacity=0.75]  (5) -- (13) -- (17) -- cycle;
	\draw[red,thick,fill=cyan!35,opacity=0.75]  (8) -- (16) -- (12) -- cycle;
	\draw[red,thick,fill=cyan!35,opacity=0.75]  (1) -- (9) -- (3) -- (12) -- (4) -- (15) -- cycle;
	\draw[red,thick,fill=cyan!35,opacity=0.75]  (0) -- (8) -- (3) -- (13) -- (5) -- (14) -- cycle;
	\draw[red,thick,fill=cyan!35,opacity=0.75]  (1) -- (7) -- (2) -- (10) -- (5) -- (17) -- cycle;
	\draw[red,thick,fill=cyan!35,opacity=0.75]  (9) -- (17) -- (13) -- cycle;
	\draw[red,thick,fill=cyan!35,opacity=0.75]  (3) -- (8) -- (12) -- cycle;
	\draw[red,thick,fill=cyan!35,opacity=0.75]  (3) -- (9) -- (13) -- cycle;
	\draw[red,thick,fill=cyan!35,opacity=0.75]  (1) -- (17) -- (9) -- cycle;
	%
	\end{tikzpicture}
	%
\end{document}



