\documentclass{article}
\usepackage{tikz}
\usepackage{tikz-3dplot}
\usepackage[active,tightpage]{preview}
\PreviewEnvironment{tikzpicture}
\setlength\PreviewBorder{0.125pt}
%
% File name: frustum.tex
% Description: 
% A geometric representation of a frustum is shown.
% 
% Date of creation: December, 12th, 2020.
% Date of last modification: August, 19th, 2023.
% Author: Efra�n Soto Apolinar.
% https://www.aprendematematicas.org.mx/author/efrain-soto-apolinar/instructing-courses/
% Source: Illustrated Dictionary of Mathematical Concepts
% https://www.aprendematematicas.org.mx/obras-distribucion-gratuita/
%
% Terms of use:
% According to TikZ.net
% https://creativecommons.org/licenses/by-nc-sa/4.0/
% Your commitment to the terms of use is greatly appreciated.
%
\begin{document}
%
\tdplotsetmaincoords{70}{120}
%
\begin{tikzpicture}[tdplot_main_coords]
	\pgfmathsetmacro{\r}{1.75}
	\pgfmathsetmacro{\h}{2.5}
	\pgfmathsetmacro{\R}{2.5}
	\pgfmathsetmacro{\n}{5}
	\pgfmathsetmacro{\previo}{\n-1}
	\pgfmathsetmacro{\anguloi}{60}
	\pgfmathsetmacro{\angulo}{360.0/\n}
	\pgfmathsetmacro{\criterio}{int(0.5*\n)}
	\pgfmathsetmacro{\pxn}{\r*cos(0.5*\anguloi)}
	\pgfmathsetmacro{\pyn}{\r*sin(0.5*\anguloi)}		
	% Calculate the coordinates of the vertices of the base
	\foreach \vertice in {1,2,...,\n}{
		\pgfmathsetmacro{\pxi}{\R*cos((\angulo)*\vertice+\anguloi)}
		\pgfmathsetmacro{\pyi}{\R*sin((\angulo)*\vertice+\anguloi)}
		\coordinate (L\vertice) at (\pxi,\pyi,0);
		\pgfmathsetmacro{\pxi}{\r*cos((\angulo)*\vertice+\anguloi)}
		\pgfmathsetmacro{\pyi}{\r*sin((\angulo)*\vertice+\anguloi)}
		\coordinate (H\vertice) at (\pxi,\pyi,\h);
	}
	% The faces of the polyhedron
	\foreach \vertice in {1,2,...,\n}{
		\pgfmathsetmacro{\vf}{\vertice+1}
		\ifthenelse{\vertice<\n}{
			\fill[cyan!50,opacity=0.5] (L\vertice) -- (0,0,0) -- (L\vf) -- (L\vertice);
			\fill[cyan!50,opacity=0.5] (L\vertice) -- (L\vf) -- (H\vf) -- (H\vertice) -- (L\vertice);
			\fill[cyan!50,opacity=0.5] (L\vertice) -- (L\vf) -- (H\vf) -- (H\vertice) -- (L\vertice);
			\draw[red,thick] (L\vertice) -- (L\vf);
			\draw[red,thick] (L\vertice) -- (H\vertice);
			\draw[red,thick] (H\vertice) -- (H\vf);
		}{ % the last face
			\fill[cyan!50,opacity=0.5] (L\vertice) -- (L1) -- (0,0,0) -- (L\vertice);
			\fill[cyan!50,opacity=0.5] (L\vertice) -- (L1) -- (H1) -- (H\vertice) -- (L\vertice);
			\fill[cyan!50,opacity=0.5] (H\vertice) -- (H1) -- (0,0,\h) -- (H\vertice);
			\fill[cyan!50,opacity=0.5] (H1) -- (H2) -- (0,0,\h) -- (H1);
			\draw[red,thick] (L\vertice) -- (L1);
			\draw[red,thick] (H\vertice) -- (H1);
			\draw[red,thick] (L\vertice) -- (H\vertice);			
		}
	}
	% Fill top face
	\foreach \vertice in {1,2,...,\previo}{
		\pgfmathsetmacro{\vf}{\vertice+1}
		\fill[cyan!50,opacity=0.5] (H\vertice) -- (0,0,\h) -- (H\vf) -- (H\vertice);
		\ifthenelse{\vertice<\n}{
			\fill[cyan!50,opacity=0.5] (H\vertice) -- (0,0,\h) -- (H\vf) -- (H\vertice);
		}{ % the last face
			\draw[red,thick] (H\n) -- (H1);
		}
	}
	\fill[cyan!50,opacity=0.5] (H\n) -- (0,0,\h) -- (H1) -- (H\n);
\end{tikzpicture}
%	
\end{document}