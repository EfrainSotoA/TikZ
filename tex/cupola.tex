\documentclass{article}
\usepackage{tikz}
\usepackage{tikz-3dplot}
\usepackage[active,tightpage]{preview}
\PreviewEnvironment{tikzpicture}
\setlength\PreviewBorder{0.125pt}
%
% File name: cupola.tex
% Description: 
% A geometric representation of a cupola is shown.
% 
% Date of creation: July, 22nd, 2023.
% Date of last modification: August, 19th, 2023.
% Author: Efraín Soto Apolinar.
% https://www.aprendematematicas.org.mx/author/efrain-soto-apolinar/instructing-courses/
% Source: Illustrated Dictionary of Mathematical Concepts
% https://www.aprendematematicas.org.mx/obras-distribucion-gratuita/
%
% Terms of use:
% According to TikZ.net
% https://creativecommons.org/licenses/by-nc-sa/4.0/
% Your commitment to the terms of use is greatly appreciated.
%
\begin{document}
\tdplotsetmaincoords{105}{180}
%
\begin{tikzpicture}[tdplot_main_coords,scale=3.0]
	%
	\pgfmathsetmacro{\cero}{(sqrt(5.0) - 1.0) / 4.0}
	\pgfmathsetmacro{\uno}{(1.0 + sqrt(5.0)) / 4.0}
	\pgfmathsetmacro{\dos}{(3.0 + sqrt(5.0)) / 4.0}
	\pgfmathsetmacro{\tres}{(1.0 + sqrt(5.0)) / 2.0}
	% Coordenadas de los vértices
	
	\coordinate(0) at (\tres, 0.0, 0.0);
	\coordinate(1) at (-\tres, 0.0, 0.0);
	\coordinate(2) at (\dos, -\uno, 0.5);
	\coordinate(3) at (\dos, \uno, -0.5);
	\coordinate(4) at (-\dos, -\uno, 0.5);
	\coordinate(5) at (-\dos, \uno, -0.5);
	\coordinate(6) at (0.5, -\dos, \uno);
	\coordinate(7) at (0.5, \dos, -\uno);
	\coordinate(8) at (-0.5, -\dos, \uno);
	\coordinate(9) at (-0.5, \dos, -\uno);
	\coordinate(10) at (\uno, 0.5, \cero);
	\coordinate(11) at (-\uno, 0.5, \cero);
	\coordinate(12) at (0.5, -\cero, \uno);
	\coordinate(13) at (-0.5, -\cero, \uno);
	\coordinate(14) at (0.0, 1.0, 0.0);
	% Faces of the polyhedron
	\draw[red,thick,fill=cyan!35,opacity=0.75]  (0) -- (2) -- (6) -- (8) -- (4) -- (1) -- (5) -- (9) -- (7) -- (3) -- cycle;
	\draw[red,thick,fill=cyan!35,opacity=0.75]  (13) -- (8) -- (6) -- (12) -- cycle;
	\draw[red,thick,fill=cyan!35,opacity=0.75]  (13) -- (4) -- (8) -- cycle;
	\draw[red,thick,fill=cyan!35,opacity=0.75]  (12) -- (6) -- (2) -- cycle;
	\draw[red,thick,fill=cyan!35,opacity=0.75]  (11) -- (1) -- (4) -- (13) -- cycle;
	\draw[red,thick,fill=cyan!35,opacity=0.75]  (14) -- (9) -- (5) -- (11) -- cycle;
	\draw[red,thick,fill=cyan!35,opacity=0.75]  (12) -- (2) -- (0) -- (10) -- cycle;
	\draw[red,thick,fill=cyan!35,opacity=0.75]  (10) -- (3) -- (7) -- (14) -- cycle;
	\draw[red,thick,fill=cyan!35,opacity=0.75]  (11) -- (5) -- (1) -- cycle;
	\draw[red,thick,fill=cyan!35,opacity=0.75]  (14) -- (7) -- (9) -- cycle;
	\draw[red,thick,fill=cyan!35,opacity=0.75]  (10) -- (0) -- (3) -- cycle;
	\draw[red,thick,fill=cyan!35,opacity=0.75]  (10) -- (14) -- (11) -- (13) -- (12) -- cycle;
	%
	\end{tikzpicture}
	%
\end{document}



