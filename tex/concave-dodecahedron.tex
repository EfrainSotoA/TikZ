\documentclass{article}
\usepackage{tikz}
\usepackage{tikz-3dplot}
\usepackage[active,tightpage]{preview}
\PreviewEnvironment{tikzpicture}
\setlength\PreviewBorder{0.125pt}
%
% File name: concave-dodecahedron.tex
% Description: 
% A geometric representation of a concave dodecahedron is shown.
% 
% Date of creation: July, 22nd, 2023.
% Date of last modification: August, 19th, 2023.
% Author: Efraín Soto Apolinar.
% https://www.aprendematematicas.org.mx/author/efrain-soto-apolinar/instructing-courses/
% Source: Illustrated Dictionary of Mathematical Concepts
% https://www.aprendematematicas.org.mx/obras-distribucion-gratuita/
%
% Terms of use:
% According to TikZ.net
% https://creativecommons.org/licenses/by-nc-sa/4.0/
% Your commitment to the terms of use is greatly appreciated.
%
\begin{document}
\tdplotsetmaincoords{85}{125}
%
\begin{tikzpicture}[tdplot_main_coords]
	%
	\pgfmathsetmacro{\escala}{3.5}
	\pgfmathsetmacro{\cero}{(sqrt(5.0) - 1.0) / 4.0}
	\pgfmathsetmacro{\uno}{(1.0 + sqrt(5.0)) / 4.0}
	% Coordenadas de los vértices
	\coordinate(0) at (\escala * 0.0, \escala * 0.5, \escala * \cero);
	\coordinate(1) at (\escala * 0.0, \escala * 0.5, -\escala * \cero);
	\coordinate(2) at (\escala * 0.0, -\escala * 0.5, \escala * \cero);
	\coordinate(3) at (\escala * 0.0, -\escala * 0.5, -\escala * \cero);
	\coordinate(4) at (\escala * \cero, \escala * 0.0, \escala * 0.5);
	\coordinate(5) at (\escala * \cero, \escala * 0.0, -\escala * 0.5);
	\coordinate(6) at (-\escala * \cero, \escala * 0.0, \escala * 0.5);
	\coordinate(7) at (-\escala * \cero, \escala * 0.0, -\escala * 0.5);
	\coordinate(8) at (\escala * 0.5, \escala * \cero, \escala * 0.0);
	\coordinate(9) at (\escala * 0.5, -\escala * \cero, \escala * 0.0);
	\coordinate(10) at (-\escala * 0.5, \escala * \cero, \escala * 0.0);
	\coordinate(11) at (-\escala * 0.5, -\escala * \cero, \escala * 0.0);
	\coordinate(12) at (\escala * \uno, \escala * \uno, \escala * \uno);
	\coordinate(13) at (\escala * \uno, \escala * \uno, -\escala * \uno);
	\coordinate(14) at (\escala * \uno, -\escala * \uno, \escala * \uno);
	\coordinate(15) at (\escala * \uno, -\escala * \uno, -\escala * \uno);
	\coordinate(16) at (-\escala * \uno, \escala * \uno, \escala * \uno);
	\coordinate(17) at (-\escala * \uno, \escala * \uno, -\escala * \uno);
	\coordinate(18) at (-\escala * \uno, -\escala * \uno, \escala * \uno);
	\coordinate(19) at (-\escala * \uno, -\escala * \uno, -\escala * \uno);
	% Faces of the polyhedron
	\draw[red,thick,fill=cyan!35,opacity=0.75]  (13) -- (8) -- (12) -- (4) -- (5) -- cycle;
	\draw[red,thick,fill=cyan!35,opacity=0.75]  (9) -- (15) -- (5) -- (4) -- (14) -- cycle;
	\draw[red,thick,fill=cyan!35,opacity=0.75]  (0) -- (12) -- (8) -- (10) -- (16) -- cycle;
	\draw[red,thick,fill=cyan!35,opacity=0.75]  (0) -- (2) -- (14) -- (4) -- (12) -- cycle;
	\draw[red,thick,fill=cyan!35,opacity=0.75]  (0) -- (16) -- (6) -- (18) -- (2) -- cycle;
	\draw[red,thick,fill=cyan!35,opacity=0.75]  (13) -- (1) -- (17) -- (10) -- (8) -- cycle;
	\draw[red,thick,fill=cyan!35,opacity=0.75]  (13) -- (5) -- (15) -- (3) -- (1) -- cycle;
	\draw[red,thick,fill=cyan!35,opacity=0.75]  (7) -- (17) -- (1) -- (3) -- (19) -- cycle;
	\draw[red,thick,fill=cyan!35,opacity=0.75]  (7) -- (6) -- (16) -- (10) -- (17) -- cycle;
	\draw[red,thick,fill=cyan!35,opacity=0.75]  (7) -- (19) -- (11) -- (18) -- (6) -- cycle;
	\draw[red,thick,fill=cyan!35,opacity=0.75]  (9) -- (11) -- (19) -- (3) -- (15) -- cycle;
	\draw[red,thick,fill=cyan!35,opacity=0.75]  (9) -- (14) -- (2) -- (18) -- (11) -- cycle;
	%
	\end{tikzpicture}
	%
\end{document}
