\documentclass{article}
\usepackage{tikz}
\usepackage{tikz-3dplot}
\usepackage[active,tightpage]{preview}
\PreviewEnvironment{tikzpicture}
\setlength\PreviewBorder{0.125pt}
%
% File name: truncated-cuboctahedron.tex
% Description: 
% A geometric representation of a truncated cuboctahedron is shown.
% 
% Date of creation: May, 16th, 2021.
% Date of last modification: August, 19th, 2023.
% Author: Efraín Soto Apolinar.
% https://www.aprendematematicas.org.mx/author/efrain-soto-apolinar/instructing-courses/
% Source: Illustrated Dictionary of Mathematical Concepts
% https://www.aprendematematicas.org.mx/obras-distribucion-gratuita/
%
% Terms of use:
% According to TikZ.net
% https://creativecommons.org/licenses/by-nc-sa/4.0/
% Your commitment to the terms of use is greatly appreciated.
%
\begin{document}
\tdplotsetmaincoords{70}{140}
%
\begin{tikzpicture}[tdplot_main_coords,scale=2.0]
	% Coordinates of the vertices
	\coordinate(UNO) at (-0.5, 1.20711, -1.91421);
	\coordinate(DOS) at (-0.5, 1.20711, 1.91421);
	\coordinate(TRES) at (-0.5, -1.20711, -1.91421);
	\coordinate(CUATRO) at (-0.5, -1.20711, 1.91421);
	\coordinate(CINCO) at (-0.5, -1.91421, 1.20711);
	\coordinate(SEIS) at (-0.5, -1.91421, -1.20711);
	\coordinate(SIETE) at (-0.5, 1.91421, 1.20711);
	\coordinate(OCHO) at (-0.5, 1.91421, -1.20711);
	\coordinate(NUEVE) at (0.5, 1.20711, -1.91421);
	\coordinate(DIEZ) at (0.5, 1.20711, 1.91421);
	%
	\coordinate(ONCE) at (0.5, -1.20711, -1.91421);
	\coordinate(DOCE) at (0.5, -1.20711, 1.91421);
	\coordinate(TRECE) at (0.5, -1.91421, 1.20711);
	\coordinate(CATORCE) at (0.5, -1.91421, -1.20711);
	\coordinate(QUINCE) at (0.5, 1.91421, 1.20711);
	\coordinate(DIECISEIS) at (0.5, 1.91421, -1.20711);
	\coordinate(DIECISIETE) at (1.20711, -0.5, -1.91421);
	\coordinate(DIECIOCHO) at (1.20711, -0.5, 1.91421);
	\coordinate(DIECINUEVE) at (1.20711, 0.5, -1.91421);
	\coordinate(VEINTE) at (1.20711, 0.5, 1.91421);
	%
	\coordinate(VEINTIUNO) at (1.20711, -1.91421, -0.5);
	\coordinate(VEINTIDOS) at (1.20711, -1.91421, 0.5);
	\coordinate(VEINTITRES) at (1.20711, 1.91421, -0.5);
	\coordinate(VEINTICUATRO) at (1.20711, 1.91421, 0.5);
	\coordinate(VEINTICINCO) at (-1.20711, -0.5, -1.91421);
	\coordinate(VEINTISEIS) at (-1.20711, -0.5, 1.91421);
	\coordinate(VEINTISIETE) at (-1.20711, 0.5, -1.91421);
	\coordinate(VEINTIOCHO) at (-1.20711, 0.5, 1.91421);
	\coordinate(VEINTINUEVE) at (-1.20711, -1.91421, -0.5);
	\coordinate(TREINTA) at (-1.20711, -1.91421, 0.5);
	%
	\coordinate(TREINTAYUNO) at (-1.20711, 1.91421, -0.5);
	\coordinate(TREINTAYDOS) at (-1.20711, 1.91421, 0.5);
	\coordinate(TREINTAYTRES) at (-1.91421, -0.5, 1.20711);
	\coordinate(TREINTAYCUATRO) at (-1.91421, -0.5, -1.20711);
	\coordinate(TREINTAYCINCO) at (-1.91421, 0.5, 1.20711);
	\coordinate(TREINTAYSEIS) at (-1.91421, 0.5, -1.20711);
	\coordinate(TREINTAYSIETE) at (-1.91421, 1.20711, -0.5);
	\coordinate(TREINTAYOCHO) at (-1.91421, 1.20711, 0.5);
	\coordinate(TREINTAYNUEVE) at (-1.91421, -1.20711, -0.5);
	\coordinate(CUARENTA) at (-1.91421, -1.20711, 0.5);
	%
	\coordinate(CUARENTAYUNO) at (1.91421, -0.5, 1.20711);
	\coordinate(CUARENTAYDOS) at (1.91421, -0.5, -1.20711);
	\coordinate(CUARENTAYTRES) at (1.91421, 0.5, 1.20711);
	\coordinate(CUARENTAYCUATRO) at (1.91421, 0.5, -1.20711);
	\coordinate(CUARENTAYCINCO) at (1.91421, 1.20711, -0.5);
	\coordinate(CUARENTAYSEIS) at (1.91421, 1.20711, 0.5);
	\coordinate(CUARENTAYSIETE) at (1.91421, -1.20711, -0.5);
	\coordinate(CUARENTAYOCHO) at (1.91421, -1.20711, 0.5);
	\draw[red,thick,fill=cyan!35,opacity=0.75]  (CINCO) -- (TREINTA) -- (VEINTINUEVE) -- (SEIS) 
			-- (CATORCE) -- (VEINTIUNO) -- (VEINTIDOS) -- (TRECE) -- (CINCO);	
	\draw[red,thick,fill=cyan!35,opacity=0.75]  (NUEVE) -- (DIECINUEVE) -- (DIECISIETE) -- (ONCE) 
			-- (TRES) -- (VEINTICINCO) -- (VEINTISIETE) -- (UNO) -- (NUEVE);
	\draw[red,thick,fill=cyan!35,opacity=0.75]  (TREINTAYCINCO) -- (TREINTAYOCHO) -- (TREINTAYSIETE) 
			-- (TREINTAYSEIS) -- (TREINTAYCUATRO) -- (TREINTAYNUEVE) -- (CUARENTA) 
			-- (TREINTAYTRES) -- (TREINTAYCINCO);
	\draw[red,thick,fill=cyan!35,opacity=0.75]  (CUARENTA) -- (TREINTA) -- (CINCO) -- (CUATRO) 
			-- (VEINTISEIS) -- (TREINTAYTRES) -- (CUARENTA);
	\draw[red,thick,fill=cyan!35,opacity=0.75]  (TREINTAYSIETE) -- (TREINTAYUNO) -- (OCHO) -- (UNO) 
			-- (VEINTISIETE) -- (TREINTAYSEIS) -- (TREINTAYSIETE);
	\draw[red,thick,fill=cyan!35,opacity=0.75]  (TREINTAYCUATRO) -- (VEINTICINCO) -- (TRES) -- (SEIS) 
			-- (VEINTINUEVE) -- (TREINTAYNUEVE) -- (TREINTAYCUATRO);
	\draw[red,thick,fill=cyan!35,opacity=0.75]  (CUARENTAYSIETE) -- (VEINTIUNO) -- (CATORCE) -- (ONCE) 
			-- (DIECISIETE) -- (CUARENTAYDOS) -- (CUARENTAYSIETE);
	\draw[red,thick,fill=cyan!35,opacity=0.75]  (TREINTAYNUEVE) -- (VEINTINUEVE) -- (TREINTA) 
			-- (CUARENTA) -- (TREINTAYNUEVE);	
	\draw[red,thick,fill=cyan!35,opacity=0.75]  (VEINTE) -- (DIECIOCHO) -- (CUARENTAYUNO) 
			-- (CUARENTAYTRES) -- (VEINTE);	
	\draw[red,thick,fill=cyan!35,opacity=0.75]  (DOCE) -- (CUATRO) -- (CINCO) -- (TRECE) -- (DOCE);
	\draw[red,thick,fill=cyan!35,opacity=0.75]  (CATORCE) -- (SEIS) -- (TRES) -- (ONCE) -- (CATORCE);
	\draw[red,thick,fill=cyan!35,opacity=0.75]  (TREINTAYCUATRO) -- (TREINTAYSEIS) -- (VEINTISIETE) 
			-- (VEINTICINCO) -- (TREINTAYCUATRO);
	\draw[red,thick,fill=cyan!35,opacity=0.75]  (CUARENTAYOCHO) -- (VEINTIDOS) -- (VEINTIUNO) 
			-- (CUARENTAYSIETE) -- (CUARENTAYOCHO);
	\draw[red,thick,fill=cyan!35,opacity=0.75]  (TREINTAYOCHO) -- (TREINTAYDOS) -- (TREINTAYUNO) 
			-- (TREINTAYSIETE) -- (TREINTAYOCHO);	
	\draw[red,thick,fill=cyan!35,opacity=0.75]  (CUARENTAYCUATRO) -- (CUARENTAYDOS) -- (DIECISIETE) 
			-- (DIECINUEVE) -- (CUARENTAYCUATRO);
	\draw[red,thick,fill=cyan!35,opacity=0.75]  (OCHO) -- (DIECISEIS) -- (NUEVE) -- (UNO) -- (OCHO);
	\draw[red,thick,fill=cyan!35,opacity=0.75]  (VEINTISEIS) -- (VEINTIOCHO) -- (TREINTAYCINCO) 
			-- (TREINTAYTRES) -- (VEINTISEIS);
	\draw[red,thick,fill=cyan!35,opacity=0.75]  (DOS) -- (DIEZ) -- (QUINCE) -- (SIETE) -- (DOS);
	\draw[red,thick,fill=cyan!35,opacity=0.75]  (CUARENTAYCINCO) -- (VEINTITRES) -- (VEINTICUATRO) 
			-- (CUARENTAYSEIS) -- (CUARENTAYCINCO);
	\draw[red,thick,fill=cyan!35,opacity=0.75]  (TREINTAYCINCO) -- (VEINTIOCHO) -- (DOS) -- (SIETE) 
			-- (TREINTAYDOS) -- (TREINTAYOCHO) -- (TREINTAYCINCO);
	\draw[red,thick,fill=cyan!35,opacity=0.75]  (CUARENTAYSEIS) -- (VEINTICUATRO) -- (QUINCE) 
			-- (DIEZ) -- (VEINTE) -- (CUARENTAYTRES) -- (CUARENTAYSEIS);
	\draw[red,thick,fill=cyan!35,opacity=0.75]  (CUARENTAYCUATRO) -- (DIECINUEVE) -- (NUEVE) -- (DIECISEIS) 
			-- (VEINTITRES) -- (CUARENTAYCINCO) -- (CUARENTAYCUATRO);
	\draw[red,thick,fill=cyan!35,opacity=0.75]  (CUARENTAYUNO) -- (DIECIOCHO) -- (DOCE) -- (TRECE) 
			-- (VEINTIDOS) -- (CUARENTAYOCHO) -- (CUARENTAYUNO);	
	\draw[red,thick,fill=cyan!35,opacity=0.75]  (DOS) -- (VEINTIOCHO) -- (VEINTISEIS) -- (CUATRO) 
			-- (DOCE) -- (DIECIOCHO) -- (VEINTE) -- (DIEZ) -- (DOS);
	\draw[red,thick,fill=cyan!35,opacity=0.75]  (CUARENTAYUNO) -- (CUARENTAYOCHO) -- (CUARENTAYSIETE) 
			-- (CUARENTAYDOS) -- (CUARENTAYCUATRO) -- (CUARENTAYCINCO) -- (CUARENTAYSEIS) 
			-- (CUARENTAYTRES) -- (CUARENTAYUNO);
	\draw[red,thick,fill=cyan!35,opacity=0.75]  (QUINCE) -- (VEINTICUATRO) -- (VEINTITRES) -- (DIECISEIS) 
			-- (OCHO) -- (TREINTAYUNO) -- (TREINTAYDOS) -- (SIETE) -- (QUINCE);
	\end{tikzpicture}
	%
\end{document}
