\documentclass{article}
\usepackage{tikz}
\usepackage{tikz-3dplot}
\usepackage[active,tightpage]{preview}
\PreviewEnvironment{tikzpicture}
\setlength\PreviewBorder{0.125pt}
%
% File name: chamfered-cube.tex
% Description: 
% A geometric representation of a chamfered cube is shown.
% 
% Date of creation: July, 15th, 2021.
% Date of last modification: August, 19th, 2023.
% Author: Efraín Soto Apolinar.
% https://www.aprendematematicas.org.mx/author/efrain-soto-apolinar/instructing-courses/
% Source: Illustrated Dictionary of Mathematical Concepts
% https://www.aprendematematicas.org.mx/obras-distribucion-gratuita/
%
% Terms of use:
% According to TikZ.net
% https://creativecommons.org/licenses/by-nc-sa/4.0/
% Your commitment to the terms of use is greatly appreciated.
%
\begin{document}
\tdplotsetmaincoords{80}{115}
%
\begin{tikzpicture}[tdplot_main_coords]
	%
	\pgfmathsetmacro{\escala}{1.25}
	\pgfmathsetmacro{\r}{\escala}
	\pgfmathsetmacro{\cero}{(3.0 + 2.0 * sqrt(3.0)) / 6.0}
	\pgfmathsetmacro{\uno}{(3.0 + 4.0 * sqrt(3.0)) / 6.0}
	% Coordinates of the vertices
	\coordinate(0) at (\escala*0.5, \escala*0.5, \escala*\uno);
	\coordinate(1) at (\escala*0.5, \escala*0.5, -\escala*\uno);
	\coordinate(2) at (\escala*0.5, -\escala*0.5, \escala*\uno);
	\coordinate(3) at (\escala*0.5, -\escala*0.5, -\escala*\uno);
	\coordinate(4) at (-\escala*0.5, \escala*0.5, \escala*\uno);
	\coordinate(5) at (-\escala*0.5, \escala*0.5, -\escala*\uno);
	\coordinate(6) at (-\escala*0.5, -\escala*0.5, \escala*\uno);
	\coordinate(7) at (-\escala*0.5, -\escala*0.5, -\escala*\uno);
	\coordinate(8) at (\escala*\uno, \escala*0.5, \escala*0.5);
	\coordinate(9) at (\escala*\uno, \escala*0.5, -\escala*0.5);
	\coordinate(10) at (\escala*\uno, -\escala*0.5, \escala*0.5);
	%
	\coordinate(11) at (\escala*\uno, -\escala*0.5, -\escala*0.5);
	\coordinate(12) at (-\escala*\uno, \escala*0.5, \escala*0.5);
	\coordinate(13) at (-\escala*\uno, \escala*0.5, -\escala*0.5);
	\coordinate(14) at (-\escala*\uno, -\escala*0.5, \escala*0.5);
	\coordinate(15) at (-\escala*\uno, -\escala*0.5, -\escala*0.5);
	\coordinate(16) at (\escala*0.5, \escala*\uno, \escala*0.5);
	\coordinate(17) at (\escala*0.5, \escala*\uno, -\escala*0.5);
	\coordinate(18) at (\escala*0.5, -\escala*\uno, \escala*0.5);
	\coordinate(19) at (\escala*0.5, -\escala*\uno, -\escala*0.5);
	\coordinate(20) at (-\escala*0.5, \escala*\uno, \escala*0.5);
	%
	\coordinate(21) at (-\escala*0.5, \escala*\uno, -\escala*0.5);
	\coordinate(22) at (-\escala*0.5, -\escala*\uno, \escala*0.5);
	\coordinate(23) at (-\escala*0.5, -\escala*\uno, -\escala*0.5);
	\coordinate(24) at (\escala*\cero, \escala*\cero, \escala*\cero);
	\coordinate(25) at (\escala*\cero, \escala*\cero, -\escala*\cero);
	\coordinate(26) at (\escala*\cero, -\escala*\cero, \escala*\cero);
	\coordinate(27) at (\escala*\cero, -\escala*\cero, -\escala*\cero);
	\coordinate(28) at (-\escala*\cero, \escala*\cero, \escala*\cero);
	\coordinate(29) at (-\escala*\cero, \escala*\cero, -\escala*\cero);
	\coordinate(30) at (-\escala*\cero, -\escala*\cero, \escala*\cero);
	%
	\coordinate(31) at (-\escala*\cero, -\escala*\cero, -\escala*\cero);
	% Faces of the polyhedron
	\draw[red,thick,fill=cyan!35,opacity=0.75]  (27) -- (19) -- (23) -- (31) -- (7) -- (3) -- cycle;
	\draw[red,thick,fill=cyan!35,opacity=0.75]  (29) -- (5) -- (7) -- (31) -- (15) -- (13) -- cycle;
	\draw[red,thick,fill=cyan!35,opacity=0.75]  (29) -- (13) -- (12) -- (28) -- (20) -- (21) -- cycle;
	\draw[red,thick,fill=cyan!35,opacity=0.75]  (30) -- (6) -- (4) -- (28) -- (12) -- (14) -- cycle;
	\draw[red,thick,fill=cyan!35,opacity=0.75]  (30) -- (14) -- (15) -- (31) -- (23) -- (22) -- cycle;
	\draw[red,thick,fill=cyan!35,opacity=0.75]  (30) -- (22) -- (18) -- (26) -- (2) -- (6) -- cycle;
	\draw[red,thick,fill=cyan!35,opacity=0.75]  (1) -- (3) -- (7) -- (5) -- cycle;
	\draw[red,thick,fill=cyan!35,opacity=0.75]  (12) -- (13) -- (15) -- (14) -- cycle;
	\draw[red,thick,fill=cyan!35,opacity=0.75]  (18) -- (22) -- (23) -- (19) -- cycle;
	%
	\draw[red,thick,fill=cyan!35,opacity=0.75]  (16) -- (17) -- (21) -- (20) -- cycle;
	\draw[red,thick,fill=cyan!35,opacity=0.75]  (29) -- (21) -- (17) -- (25) -- (1) -- (5) -- cycle;
	\draw[red,thick,fill=cyan!35,opacity=0.75]  (24) -- (8) -- (9) -- (25) -- (17) -- (16) -- cycle;
	\draw[red,thick,fill=cyan!35,opacity=0.75]  (27) -- (3) -- (1) -- (25) -- (9) -- (11) -- cycle;
	\draw[red,thick,fill=cyan!35,opacity=0.75]  (8) -- (10) -- (11) -- (9) -- cycle;
	\draw[red,thick,fill=cyan!35,opacity=0.75]  (27) -- (11) -- (10) -- (26) -- (18) -- (19) -- cycle;
	\draw[red,thick,fill=cyan!35,opacity=0.75]  (24) -- (16) -- (20) -- (28) -- (4) -- (0) -- cycle;
	\draw[red,thick,fill=cyan!35,opacity=0.75]  (24) -- (0) -- (2) -- (26) -- (10) -- (8) -- cycle;
	\draw[red,thick,fill=cyan!35,opacity=0.75]  (0) -- (4) -- (6) -- (2) -- cycle;
	%
	\end{tikzpicture}
	%
\end{document}



