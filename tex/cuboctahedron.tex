\documentclass{article}
\usepackage{tikz}
\usepackage{tikz-3dplot}
\usepackage[active,tightpage]{preview}
\PreviewEnvironment{tikzpicture}
\setlength\PreviewBorder{0.125pt}
%
% File name: cuboctahedron.tex
% Description: 
% A geometric representation of a cuboctahedron is shown.
% 
% Date of creation: May, 25th, 2020.
% Date of last modification: August, 19th, 2023.
% Author: Efraín Soto Apolinar.
% https://www.aprendematematicas.org.mx/author/efrain-soto-apolinar/instructing-courses/
% Source: Illustrated Dictionary of Mathematical Concepts
% https://www.aprendematematicas.org.mx/obras-distribucion-gratuita/
%
% Terms of use:
% According to TikZ.net
% https://creativecommons.org/licenses/by-nc-sa/4.0/
% Your commitment to the terms of use is greatly appreciated.
%
\begin{document}
\tdplotsetmaincoords{70}{120}
%
\begin{tikzpicture}[tdplot_main_coords]
	%	
	\pgfmathsetmacro{\escala}{2.5}
	\pgfmathsetmacro{\a}{\escala*sqrt(2.0)}
	\pgfmathsetmacro{\b}{\escala*1.0/sqrt(2.0)}
	
	% Coordenadas de los vértices
	\coordinate(UNO) at (-1*\escala,0,0);
	\coordinate(DOS) at (-0.5*\escala,-0.5*\escala,-\b);
	\coordinate(TRES) at (-0.5*\escala,-0.5*\escala,\b);
	\coordinate(CUATRO) at (-0.5*\escala,0.5*\escala,-\b);
	\coordinate(CINCO) at (-0.5*\escala,0.5*\escala,\b);
	\coordinate(SEIS) at (0,-1*\escala,0);
	\coordinate(SIETE) at (0,1*\escala,0);
	\coordinate(OCHO) at (0.5*\escala,-0.5*\escala,-\b);
	\coordinate(NUEVE) at (0.5*\escala,-0.5*\escala,\b);
	\coordinate(DIEZ) at (0.5*\escala,0.5*\escala,-\b);
	\coordinate(ONCE) at (0.5*\escala,0.5*\escala,\b);
	\coordinate(DOCE) at (1*\escala,0,0);
	% Faces of the polyhedron
	\draw[red,thick,fill=cyan!35,opacity=0.75] (DOCE) -- (ONCE) -- (NUEVE) -- (DOCE);
	\draw[red,thick,fill=cyan!35,opacity=0.75] (CINCO) -- (ONCE) -- (SIETE) -- (CINCO);
	\draw[red,thick,fill=cyan!35,opacity=0.75] (OCHO) -- (DIEZ) -- (DOCE) -- (OCHO);
	\draw[red,thick,fill=cyan!35,opacity=0.75] (SIETE) -- (DIEZ) -- (CUATRO) -- (SIETE);
	% 
	\draw[red,thick,fill=cyan!35,opacity=0.75] (ONCE) -- (DOCE) -- (DIEZ) -- (SIETE) -- (ONCE);
	\draw[red,thick,fill=cyan!35,opacity=0.75] (NUEVE) -- (SEIS) -- (OCHO) -- (DOCE) -- (NUEVE);
	\draw[red,thick,fill=cyan!35,opacity=0.75] (TRES) -- (NUEVE) -- (ONCE) -- (CINCO) -- (TRES);
	\draw[red,thick,fill=cyan!35,opacity=0.75] (CINCO) -- (SIETE) -- (CUATRO) -- (UNO) -- (CINCO);
	\draw[red,thick,fill=cyan!35,opacity=0.75] (TRES) -- (UNO) -- (DOS) -- (SEIS) -- (TRES);
	\draw[red,thick,fill=cyan!35,opacity=0.75] (CUATRO) -- (DIEZ) -- (OCHO) -- (DOS) -- (CUATRO);
	% 
	\draw[red,thick,fill=cyan!35,opacity=0.75] (TRES) -- (CINCO) -- (UNO) -- (TRES);
	\draw[red,thick,fill=cyan!35,opacity=0.75] (SEIS) -- (NUEVE) -- (TRES) -- (SEIS);
	\draw[red,thick,fill=cyan!35,opacity=0.75] (UNO) -- (CUATRO) -- (DOS) -- (UNO);
	\draw[red,thick,fill=cyan!35,opacity=0.75] (DOS) -- (OCHO) -- (SEIS) -- (DOS);
	\end{tikzpicture}
\end{document}